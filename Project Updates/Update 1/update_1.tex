\documentclass[12pt]{article}

\usepackage{fancyhdr}
\usepackage{extramarks}
\usepackage{amsmath}
\usepackage{amsthm}
\usepackage{amsfonts}
\usepackage{tikz}
\usepackage{graphicx}
\usepackage{enumitem}
\usepackage[plain]{algorithm}
\usepackage{algpseudocode}
\usepackage[normalem]{ulem}
\usepackage{hyperref}
\usepackage{xcolor}
\usepackage{caption}
\usepackage{subcaption}
\usepackage{listings}% http://ctan.org/pkg/listings
\lstset{
  basicstyle=\ttfamily,
  mathescape
}

\hypersetup{
    colorlinks=true,
    linkcolor=blue,
    filecolor=magenta,
    urlcolor=blue,
}

\usetikzlibrary{automata,positioning}

\topmargin=-0.45in
\evensidemargin=0in
\oddsidemargin=0in
\textwidth=6.5in
\textheight=9.0in
\headsep=0.25in

\linespread{1.1}

\pagestyle{fancy}
\lhead{\hmwkAuthorName}
\rhead{\hmwkClass\: \hmwkTitle}
% \rhead{\firstxmark}
\lfoot{\lastxmark}
\cfoot{\thepage}

\renewcommand\headrulewidth{0.4pt}
\renewcommand\footrulewidth{0.4pt}

\setlength\parindent{0pt}

\setcounter{secnumdepth}{0}
\newenvironment{homeworkProblem}[1][-1]{
    \ifnum#1>0
        \setcounter{homeworkProblemCounter}{#1}
    \fi
    \section{Problem \arabic{homeworkProblemCounter}}
    \setcounter{partCounter}{1}
    \enterProblemHeader{homeworkProblemCounter}
}{
    \exitProblemHeader{homeworkProblemCounter}
}

\newcommand{\hmwkTitle}{Project Update 1}
\newcommand{\hmwkClass}{CSE 576 (Sp16)}
\newcommand{\hmwkAuthorName}{Ryan Drapeau | Sonja Khan | Aaron Nech}
\newcommand{\hmwkAuthorCSE}{\{drapeau, sonjak3, necha\}@cs.washington.edu}

\title{
    \vspace{2in}
    \textmd{\textbf{\hmwkClass:\ \hmwkTitle}}\\
    \vspace{0.1in}\large{\textit{\hmwkClassInstructor\ \hmwkClassTime}}\\
    \author{\textbf{\hmwkAuthorName\ $\vert$ \hmwkAuthorCSE\ $\vert$ \hmwkAuthorId}}
}

\date{}

\renewcommand{\part}[1]{\textbf{\large Part \Alph{partCounter}}\stepcounter{partCounter}\\}

\DeclareMathOperator*{\argmin}{arg\,min}

\DeclareMathOperator*{\argmax}{arg\,max}

\begin{document}

\section{This Week's Progress}
\vspace{-3pt}
We met with Garrett Genereux, climbing coordinator, who manages the \href{https://www.washington.edu/ima/uwild/crags-climbing-center/}{Crags Climbing Center} at UW and proposed our project idea and needs. Garrett was very open and agreed to give us off-hours access to the climbing center in order to conduct our research and project. We also proposed having an open ``CV/AR climbing night'' towards the end of the quarter where people could come in and try out our game/project. This will allow us to have uninterrupted development time when the gym is closed to gather images, videos, and test our program.\\[-5pt]

We reserved a projector from the \href{http://www.cte.uw.edu/STFEquipment}{University of Washington Student Technology Fund (STF)} to borrow for the duration of the project. We also acquired a Logitech webcam to use as our image/video source for the project. The camera records 640x480 resolution, which if it turns out to be insufficient quality, we have a Mac laptop which has a built in webcam recording at 1280x720 resolution. We will experiment with both and see which will produce high quality results / data for us to use.
\vspace{-15pt}
\section{Outline / Milestones}
\vspace{-3pt}
We will start by building an image processing API to support our setup and provide the foundations for developing games.\\[-5pt]

\textbf{Rock Wall Image Processing API}
\begin{itemize}[noitemsep]
    \item Obtain image samples of walls (simple camera photographs: maybe some with webcams / laptop cameras)
    \item Compute location of handles accurately and reliably. Ideas:
    \begin{itemize}[noitemsep]
        \item Sum of gradient magnitudes
        \item Distribution of gradient magnitudes
        \item Grey (wall) subtraction
        \item \href{http://docs.opencv.org/2.4/modules/objdetect/doc/objdetect.html}{Open CV Object Detection}
    \end{itemize}
    \item Detect Human (Histogram of Gradient human detector)
    \item Pose Detection
    \begin{itemize}[noitemsep]
        \item Many possible avenues, but some implementations use neural network pose estimation and HoG descriptors regressing to some constraints
        \item \href{http://groups.inf.ed.ac.uk/calvin/articulated_human_pose_estimation_code/}{Example Code}
    \end{itemize}
    \item Display images on the rock wall with little to no distortion
    \begin{itemize}[noitemsep]
        \item Compute incline of wall and transform image accordingly
        \item Project a dot-grid on the wall and compute the relative depth (warping) via dot pixel size, apply inverse to image being projected. Or use a depth camera.
    \end{itemize}
    \item Higher level APIs like: getHandlePositions(), didHumanTouchHandle(), didTouchXY(), trackHandleProgress()
\end{itemize}

\end{document}
